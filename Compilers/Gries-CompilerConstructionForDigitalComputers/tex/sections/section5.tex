\documentclass[../main.tex]{subfiles}

\begin{document}
\begin{enumerate}[(a)]

\item ¿Cuál es la probabilidad de que el número de piezas procesadas en 1
hora por la estación de trabajo 1 sea menor a 7?

Las variables aleatorias son:

$$T_o: \ \text{Piezas rpocesadas en la estación}$$
$$M_1: \ \text{Número de piezas procesadas con la máquina 1 en una hora}$$
$$M_2: \ \text{Número de piezas procesadas con la máquina21 en una hora}$$

$$M_1 \sim Poisson(x; \lambda = \lambda_1 = 5, t = 1) \wedge M_2 \sim Poisson(x; \lambda = \lambda_2 = 3, t = 1)$$
$$T_0 = M_1 + M_2 \Rightarrow M_{T_{0}}(t) = M_{M_{1}}(t) + M_{M_{2}}(t)$$
$$M_{T_{0}}(t) = M_{M_{1}}(t) + M_{M_{2}}(t) = e^{(\lambda_1 + \lambda_2) \times (e^t - 1)}$$

$$\Rightarrow T_o \sim Poisson(x; \lambda = \lambda_1 + \lambda_2; t = 1) = \frac{e^{-t(\lambda_1 + \lambda_2)}(t(\lambda_1 + \lambda_2)^x)}{x!}$$

$$\mathbb{P}(X < 7) = \sum_{x = 0}^{6} \left( \frac{e^{-t(\lambda_1 + \lambda_2)}(t(\lambda_1 + \lambda_2)^x)}{x!}\right) = 0.31337$$

\item Si cada operario verificó una pieza en la estación 2, para un total de 3
piezas revisadas ¿cuál es la probabilidad de que a lo sumo dos de dichas piezas
analizadas tengan que ser desechadas?

Las variables aleatorias son:

$$R: \ \text{Número de piezas defectuosas halladas por uno de los tres operadores}$$
$$E: \ \text{Número de piezas defectuosas halladas en 2 horas}$$

$$R \sim B(x; p) = p^x(1-p)^{1-x} \wedge E \sim Binom(x; k, p) = \binom{k}{x}p^xq^{k - x}$$

Funciones generadoras de momento:

$$M_{x}(t) = 1 - p + pe^t \wedge M_{x}(t) = (1 - p + pe^t)^x$$

Probabilidad de una pieza defectuosa:

$$p = 1 - 0.85 = 0.15 \Rightarrow R_{i}(x; p = 0.15) \wedge E = \sum_{i = 1}^{3}R_{i} \Rightarrow M_{E}(t) = \prod_{i  = 1}^{3} M_{R_{i}}(t)$$
$$M_{E}(t) = (1 - p + pe^t)^3 \Rightarrow E \sim Binom(x; n = 3, p = 0.15) = \binom{k}{x}p^xq^{k - x}$$

$$\mathbb{P}(E < 2) = \sum_{x = 0}^{2} \left[ Binom(x; n = 3, p = 0.15) \right] = 0.99662$$

La probabilidad de que máximo dos piezas deban ser desechadas es $0.99662$.

\item Si se sabe que cada operario realiza el proceso de control de calidad
revisando una a una las piezas hasta encontrar la primera pieza defectuosa ¿Cuál
es la probabilidad de que la quinceava pieza revisada de forma independiente por
cualquiera de los tres operarios sea la tercera pieza no defectuosa que se encuentra
en la producción?

Las variables aleatorias son:

$N: $ Número de piezas defectuosas revisadas hasta encontrar la tercera defectuosa.\\
$N_{i}: $ Número de piezas revisadas por el operario i, hasta encontrar la primera pieza defectuosa.

$$N \sim Bin^*(x; p) = \binom{x - 1}{k - 1}q^{x - k}p^k \wedge N_{i} \sim Geom(x; p) = pq^{1 - x}$$

$$M_{N}(t) = \left( \frac{p}{-qe^t + 1} \right) ^ k \wedge M_{N_i}(t) = \frac{p}{1-qe^t}$$

Se busca la distribución de la variable aleatoria $E$.

$$E = \sum_{i = 1}^{3} N_{i} \Rightarrow M_{E}(t) = \prod_{i = 1}^{3} M_{N_i}(t) = \left( \frac{0.15}{1-0.85e^t} \right)^3$$

$$\Rightarrow E \sim Bin^*(x; k = 3, p = 0.15) = \binom{x - 1}{2}0.85^{x - 3}0.15^3$$

$$\mathbb{P}(X = 15) = \binom{14}{2}0.85^{x - 3}0.15^3 = 0.175$$

La probabilidad de que la tercera pieza no defectuosa sea la número 15 en ser revisada es: $7.2509 \times 10^{-9}$

\end{enumerate}
\end{document}
